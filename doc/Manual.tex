\documentclass[a4paper,12pt]{article}

% --------------------------------------------------------------------------- %
\usepackage[utf8]{inputenc}             % Encodamento utf-8
\usepackage[brazil]{babel}              % Parcote para texto em português

\usepackage{setspace}                   % Espaçamento flexível
\usepackage{indentfirst}                % Indentação do primeiro parágrafo
\usepackage[fixlanguage]{babelbib}      % Opções extras para linguagem

\usepackage[usenames,svgnames,dvipsnames]{xcolor}
\usepackage[font=small,format=plain,labelfont=bf,up,textfont=it,up]{caption}
\usepackage[a4paper,top=2.54cm,bottom=2.0cm,left=2.0cm,right=2.54cm]{geometry}

\usepackage[pdftex,plainpages=false,pdfpagelabels,pagebackref,colorlinks=true,
            citecolor=DarkGreen,linkcolor=NavyBlue,urlcolor=DarkCornflowerBlue,
            filecolor=green,bookmarksopen=true]{hyperref}   % links coloridos
\usepackage[all]{hypcap}                % soluciona o problema com o hyperref 
                                        % e capitulos

\newpage %%%%%%%%%%%%%%%%%%%%%%%%%%%%%%%%%%%%%%%%%%%%%%%%%%%%%%%%%%%%%%%%%%%%%%%
\pagenumbering{arabic} % começamos a numerar 
\begin{document}


\begin{center} 
  {\Huge \textcolor{NavyBlue}{ \textbf{Documentação}}}
\end{center}

\bigskip
\bigskip
\bigskip



\begin{center} 
  {\Huge \textcolor{NavyBlue}{ \textbf{Jogo das canoas}}}
\end{center}


\newpage %%%%%%%%%%%%%%%%%%%%%%%%%%%%%%%%%%%%%%%%%%%%%%%%%%%%%%%%%%%%%%%%%%%%%%%

{\LARGE \textcolor{NavyBlue}{ \textbf{Índice:}}}
\bigskip
\bigskip
\bigskip

{\large \textcolor{CornflowerBlue}{\textbf{Bibliotecas (Arquivos .h)}}}
\begin{verbatim}

allegro.h . . . . . . . . . . . pág. 03
getopt.h  . . . . . . . . . . . pág. 03
statistical.h . . . . . . . . . pág. 03
strip.h . . . . . . . . . . . . pág. 03
terrain.h . . . . . . . . . . . pág. 03
test.h. . . . . . . . . . . . . pág. 03
utils.h . . . . . . . . . . . . pág. 03

\end{verbatim}

{\textcolor{NavyBlue}{ LIST: }}
\begin{verbatim}
list.h . . . . . . . . . . . pág. 04
list-internal.h. . . . . . . pág. 04
list-item.h. . . . . . . . . pág. 04

\end{verbatim}

{\textcolor{NavyBlue}{RIVER:}}
\begin{verbatim}
river.h . . . . . . . . . . pág. 04
river-internal.h. . . . . . pág. 04





\end{verbatim}

{\large \textcolor{CornflowerBlue}{\textbf{Arquivos .c}}}
\begin{verbatim}
allegro.c . . . . . . . . . . . pág. 05
getopt.c. . . . . . . . . . . . pág. 05
list.c. . . . . . . . . . . . . pág. 06
river . . . . . . . . . . . . . pág. 07
statistical.c . . . . . . . . . pág. 07
strip.c . . . . . . . . . . . . pág. 08
test.c. . . . . . . . . . . . . pág. 08
utils.c . . . . . . . . . . . . pág. 08

\end{verbatim}


\newpage %%%%%%%%%%%%%%%%%%%%%%%%%%%%%%%%%%%%%%%%%%%%%%%%%%%%%%%%%%%%%%%%%%%%%%%%%%%%%%%%%%%%%%%%%%%%%%%%%%%%%%%%%%%%%%%%%%%%%%%%

{\Large \textcolor{NavyBlue}{ \textbf{Bibliotecas  }}}{\large \textcolor{CornflowerBlue}{  allegro.h}}
\bigskip

Cria as funções que são responsáveis pela imagem gerada em interface gráfica como as que
desenham quadrados de água, de terra e a elipse que representa o barco. Além de chamar a
biblioteca allegro e abrir e fechar a janela. 
\bigskip
\bigskip
\bigskip

{\Large \textcolor{NavyBlue}{ \textbf{Bibliotecas  }}}{\large \textcolor{CornflowerBlue}{  getopt.h}}
\bigskip

Cria a função {\textcolor{CornflowerBlue}{char getopt(int argc, char **argv, char *optstring);}}
que controla o menu de opções. Através dele, recebemos as opções que o usuário entra como parâmetro e 
e a utilizamos. Também cria variáveis auxiliares à respectiva função.
\bigskip
\bigskip
\bigskip

{\Large \textcolor{NavyBlue}{ \textbf{Bibliotecas  }}}{\large \textcolor{CornflowerBlue}{  statistical.h}}
\bigskip

Cria as funções estatísticas
{\textcolor{CornflowerBlue}{void  stat\_set\_seed(int seed);}},  
{\textcolor{CornflowerBlue}{float stat\_gen\_uniform\_float(float init, float end);}} e 
{\textcolor{CornflowerBlue}{int stat\_gen\_uniform\_int(int init, int end);}} no módulo 
{\textcolor{CornflowerBlue}{statistical.c}} que auxiliam na geração aleatória do rio e suas ilhas.
\bigskip
\bigskip
\bigskip

{\Large \textcolor{NavyBlue}{ \textbf{Bibliotecas  }}}{\large \textcolor{CornflowerBlue}{  strip.h}}
\bigskip

Responsável pelas funções que manipulam as faixas de terra do terreno com as funções
{\textcolor{CornflowerBlue}{void   tstrip\_seed(int);}}, 
{\textcolor{CornflowerBlue}{void   tstrip\_free(TStrip);}}, 
{\textcolor{CornflowerBlue}{void   tstrip\_print(TStrip);}}, 
{\textcolor{CornflowerBlue}{void   tstrip\_island(float prob, int freq);}} e 
{\textcolor{CornflowerBlue}{TStrip tstrip\_generate(int size, int zone, float normalization, TStrip base);}} que estão localizadas no módulo 
{\textcolor{CornflowerBlue}{strip.c}}.
\bigskip
\bigskip
\bigskip

{\Large \textcolor{NavyBlue}{ \textbf{Bibliotecas  }}}{\large \textcolor{CornflowerBlue}{  terrain.h}}
\bigskip

Responsável pelas funções que controlam a configuração do terreno como a velocidade e tipo, no caso terra ou água.
\bigskip
\bigskip
\bigskip

{\Large \textcolor{NavyBlue}{ \textbf{Bibliotecas  }}}{\large \textcolor{CornflowerBlue}{  test.h}}
\bigskip

É responsável pelas funções que realizam os testes com velocidades, margens e fluxos de acordo com o número de frames/iterações que o usuário escolher.
Serão passados os valores de acordo com o n-ésimo frame.

\bigskip
\bigskip
\bigskip

{\Large \textcolor{NavyBlue}{ \textbf{Bibliotecas  }}}{\large \textcolor{CornflowerBlue}{  utils.h}}
\bigskip

Responsável pela função de alocação de memória de uma forma que o programa será finalizado caso haja erro de alocação.


\newpage %%%%%%%%%%%%%%%%%%%%%%%%%%%%%%%%%%%%%%%%%%%%%%%%%%%%%%%%%%%%%%%%%%%%%%%%%%%%%%%%%%%%%%%%%%%%%%%%%%%%%%%%%%%%%%%%%%%%%

{\Large \textcolor{NavyBlue}{ \textbf{Bibliotecas  }}}{\large \textcolor{CornflowerBlue}{  LIST  list.h}}
\bigskip

Cria as funções de manipulação de listas ligadas circulares com cabeça duplamente encadeadas. Tais como remoção, inicialização, inserção, seleção, entre outros.
\bigskip
\bigskip
\bigskip

{\Large \textcolor{NavyBlue}{ \textbf{Bibliotecas  }}}{\large \textcolor{CornflowerBlue}{  LIST  list-internal.h}}
\bigskip

É responsável pela configuração da struct que manipulamos em list.h
\bigskip
\bigskip
\bigskip

{\Large \textcolor{NavyBlue}{ \textbf{Bibliotecas  }}}{\large \textcolor{CornflowerBlue}{  LIST  list-item.h}}
\bigskip

Define um vetor LItem(*LItem) como um terrain.
\bigskip
\bigskip
\bigskip

{\Large \textcolor{NavyBlue}{ \textbf{Bibliotecas  }}}{\large \textcolor{CornflowerBlue}{  RIVER  river.h}}
\bigskip

Controla as funções que irão manipular a configuração do rio.
\bigskip
\bigskip
\bigskip

{\Large \textcolor{NavyBlue}{ \textbf{Bibliotecas  }}}{\large \textcolor{CornflowerBlue}{  RIVER  river-internal.h}}
\bigskip

É responsável pela struct de configuração do rio, controlando o fluxo, altura, largura, zona, probabilidade e frequência.


\newpage %%%%%%%%%%%%%%%%%%%%%%%%%%%%%%%%%%%%%%%%%%%%%%%%%%%%%%%%%%%%%%%%%%

{\Large \textcolor{NavyBlue}{ \textbf{Arquivos .c  }}}{\large \textcolor{CornflowerBlue}{  allegro.c}}
\bigskip

Manipula as funções que são responsáveis pela imagem gerada na tela e pela criação e finalização da janela através
das funções:

{\textcolor{CornflowerBlue}{void gui\_init(void)}} Inicializa a biblioteca allegro. 

{\textcolor{CornflowerBlue}{void gui\_window\_clear(void)}} Pinta a janela inteira com a cor preta (0, 0, 0).  

{\textcolor{CornflowerBlue}{void gui\_window\_update(void)}} Carrega a janela, atualiza as configurações que foram feitas.
 
{\textcolor{CornflowerBlue}{void gui\_window\_create(int length, int height)}} Cria uma janela de altura height e altura length. 

{\textcolor{CornflowerBlue}{int gui\_window\_destroy(void)}} Fecha a janela e finaliza o programa quando o usuário clica em 'X'.

{\textcolor{CornflowerBlue}{void gui\_river\_create\_land(float x1, float y1)}} Desenha quadrados na cor da terra nas 
coordenas (x1, y1).

{\textcolor{CornflowerBlue}{void gui\_river\_create\_water(float x1, float y1)}} Desenha quadrados na cor da água nas
coordenadas (x1, y1). 

{\textcolor{CornflowerBlue}{void gui\_river\_create\_margin(int x1, int y1, int x2, int y2, int x3, int y3)}} Desenha triangulos 
nas cores de terra e água, formando a margem. Essa função faz triangulos com vértices de coordenas (x1, y1), (x2, y2) e (x3, y3) 
em cor de terra, e como esses triangulos sempre serão metade de quadrados, desenhamos na outra metade do mesmo quadrado,
 outro triangulo com cor de água.

{\textcolor{CornflowerBlue}{void gui\_boat\_create(float x, float y)}} Desenha uma canoa em forma de elipse, com centro em (x, y).


\bigskip
\bigskip
\bigskip
\bigskip
\bigskip

{\Large \textcolor{NavyBlue}{ \textbf{Arquivos .c  }}}{\large \textcolor{CornflowerBlue}{  getopt.c}}
\bigskip

Este módulo possui a função getopt recebe todos os parâmetros passados para o main pelo usuário, a quantidade de palavras recebidas(arg c) e também, um vetor de chars(optstring) que possui todas as opções de letras possíveis, sendo que este é apontado por um inteiro que controlará os valores já verificados... Como o L (de Largura), o H(de altura), entre outros. Essa função compara, a cada iteração, os valores recebidos pelo main e os valores de optstring... Caso seja encontrado um valor semelhante e ele tenha o "-" antes, esse char é devolvido pela função, ele é movido para o início pela função {\textcolor{CornflowerBlue}{move\_to\_init(char **argv, int initial\_pos);}} (função que se localiza neste mesmo módulo) e o inteiro que aponta para aoptstring passa a apontar para o primeiro valor ainda não verificado.

\bigskip
\bigskip
\bigskip

\newpage %%%%%%%%%%%%%%%%%%%%%%%%%%%%%%%%%%%%%%%%%%%%%%%%%%%%%%%%%%%%%%%%%%%%%%%%%%%%%%%%%%%%%%%%%%%%%%%%%%%%%%%%%%%%%%

\bigskip
\bigskip

{\Large \textcolor{NavyBlue}{ \textbf{Arquivos .c  }}}{\large \textcolor{CornflowerBlue}{  list.c}}
\bigskip

Este módulo irá controlar as listas ligadas circulares com cabeça duplamente encadeadas de uma forma generalizada, permitindo que tipos diferentes de estruturas possam utilizar esse meio de manipulação de lista.
Nele estão localizadas várias funções como...

{\textcolor{CornflowerBlue}{List list\_init(int N)}} inicializa a lista, criando uma lista e uma cabeça para ela... E como é uma lista circular duplamente encadeada, atribuímos a cabeça como o próximo e o anterior dela mesma.

{\textcolor{CornflowerBlue}{void list\_free(List list)}} libera a lista inteira, inclusive a cabeça. Essa função é de suma importância, pois desocupa os espaços que serão mais utilizados na memória.

{\textcolor{CornflowerBlue}{int list\_empty(List list)}} retorna 1 caso a lista esteja vazia e 0 caso possua algum elemento.
Observe que uma lista vazia é aquela que possui apenas a cabeça. 

{\textcolor{CornflowerBlue}{LItem list\_remove(List list, Link node)}} irá remover uma determinada célula da lista e irá retornar o valor(item) que ela possuia.
Quando essa célula é removido, a célula posterior passará a ser adjacente à anterior e a célula removida passará a ser excluída da memória usando free.

{\textcolor{CornflowerBlue}{void list\_insert(List list, LItem item)}} adiciona uma célula à lista de forma que esta seja adicionada após a cabeça. Ou seja, cabeça->próximo = cálula. 

{\textcolor{CornflowerBlue}{Link  list\_head(List list)}} devolve o endereço da cabeça da lista, passada como parâmetro.

{\textcolor{CornflowerBlue}{Link  list\_next(Link node)}} devolve o endereço da célula seguinte à célula que foi passada como parâmetro.

{\textcolor{CornflowerBlue}{Link  list\_prev(Link node)}} devolve o endereço da célula anterior à que foi passada como parâmetro.

{\textcolor{CornflowerBlue}{LItem list\_item(Link node)}} retorna o item que está contido na célula que foi passada como parâmetro.

{\textcolor{CornflowerBlue}{void list\_select(List list, int direction, void (*visit) (LItem))}} Percorrerá a lista de frente 
para trás caso direction seja igual a HEAD, caso contrário, percorrerá a lista de trás para frente.

{\textcolor{CornflowerBlue}{Link list\_new\_node(LItem item)}} cria uma nova célula preenchida por um item passado como parâmetro. Essa nova célula criada não se liga a nenhuma outra, é uma célula inicialmente isolada.

\newpage  %%%%%%%%%%%%%%%%%%%%%%%%%%%%%%%%%%%%%%%%%%%%%%%%%%%%%%%%%%%%%%%%%%%%%%%%%%%%%%%%%%%%%%%%%%%%%%%

{\Large \textcolor{NavyBlue}{ \textbf{Arquivos .c  }}}{\large \textcolor{CornflowerBlue}{  river.c}}
\bigskip

Esse módulo irá controlar as características do rio, como o fluxo, a velocidade de cada ponto, entre outros com as seguintes funções:

{\textcolor{CornflowerBlue}{void river\_config\_flux(float flux)}} recebe como parâmetro um valor e o passa para o atribui ao fluxo do rio.

{\textcolor{CornflowerBlue}{void river\_config\_island (float prob\_island, int freq\_island)}} atribui a probabilidade de surgimento de uma ilha e a distância mínima que ela deve estar de uma outra ilha ao rio, a partir dos parâmetros recebidos.

{\textcolor{CornflowerBlue}{void river\_config\_size(int length, int height)}} atribui altura e largura ao rio a partir dos parâmetros recebidos. 

{\textcolor{CornflowerBlue}{void river\_config\_margins(int zone)}} atribui uma zona de conforto ao rio que é a mínima distância que uma margem pode ter da outra, impedindo que o rio fique inavegável.

{\textcolor{CornflowerBlue}{void river\_animation\_generate(int seed)}} é responsável por gerar o rio, ela irá chamar outras funções que atribuirão estatisticamente valores para iniciar o rio e que preencherão o rio com h linhas
que contêm informações de ser água ou terra e velocidade. Sendo h a altura do rio atribuída pelo usuário.

{\textcolor{CornflowerBlue}{void river\_animation\_iterate()}} é responsável pela imagem que o usuário verá na tela, pois pegará o valor da altura e juntamente com a função {\textcolor{CornflowerBlue}{void river\_animation\_generate(int seed)}}, gerará um frame do rio, juntando as h faixas de terreno do repectivo frame.

{\textcolor{CornflowerBlue}{void strip\_print(TStrip strip)}} irá imprimir uma determinada linha do rio em interface gráfica 
chamando as funções de allegro.h. Nessa função será verificado se determinado pixel é terra, água ou margem para assim poder
usar a biblioteca allegro através de outas funções.

{\textcolor{CornflowerBlue}{void river\_animation\_finish()}} é responsável por finalizar o rio, desalocando as células utilizadas por ele.

\bigskip
\bigskip
\bigskip
\bigskip
\bigskip

{\Large \textcolor{NavyBlue}{ \textbf{Arquivos .c  }}}{\large \textcolor{CornflowerBlue}{  statistical.c}}
\bigskip

Esse módulo é responsável por gerar números aleatórios de acordo com determinadas distribuições que são determinadas pelas seguintes funções:

{\textcolor{CornflowerBlue}{void stat\_set\_seed(int seed)}} inicializa a semente dos números randomicos.

{\textcolor{CornflowerBlue}{float stat\_gen\_uniform\_float(float min, float max)}} gera números aleatórios do tipo float entre min e max de acordo com a distribuição uniforme.

{\textcolor{CornflowerBlue}{int stat\_gen\_uniform\_int(int min, int max)}} gera números aletórios do tipo inteiros entre min e max (incluindo min e max) de acordo com a distribuição uniforme

\newpage %%%%%%%%%%%%%%%%%%%%%%%%%%%%%%%%%%%%%%%%%%%%%%%%%%%%%%%%%%%%%%%%%%%%%%%%%%%%%%%%%%%%%%%%%%%%%%%%%%%%%%%


{\Large \textcolor{NavyBlue}{ \textbf{Arquivos .c  }}}{\large \textcolor{CornflowerBlue}{  strip.c}}
\bigskip

Manipula as faixas de terra, que são as linhas da imagem, com as seguintes funções:

{\textcolor{CornflowerBlue}{void   tstrip\_seed(int)}} irá inicializar a seed através de uma função no módulo estístico.

{\textcolor{CornflowerBlue}{void tstrip\_island(float prob, int freq)}} irá atribuir valores recebidos por parâmetros, à variáveis internas de frequência e probabilidade de surgimento de ilha.

{\textcolor{CornflowerBlue}{void tstrip\_free(TStrip strip)}} irá remover a faixa de terra utilizando a função de list.c, desse modo, esse espaço será desalocado da memória.

{\textcolor{CornflowerBlue}{TStrip tstrip\_generate(int size, int zone, float normalization, TStrip base)}} gera uma faixa de terra. Se ela for a primeira linha, esta será gerada unicamente com funções estísticas, porém, se ela não for a primeira faixa, irá gerar a linha estatisticamente de acordo com a linha anterior... Também atribuirá velocidades aos pontos de acordo com funções estatísticas, sempre tomando como base a faixa de terra anterior. 

\bigskip
\bigskip
\bigskip
\bigskip
\bigskip

{\Large \textcolor{NavyBlue}{ \textbf{Arquivos .c  }}}{\large \textcolor{CornflowerBlue}{  test.c}}
\bigskip

{\textcolor{CornflowerBlue}{void analyse\_program(int seed, int iterations)}} fará com que o programa gere um rio com n iterações para que na n-ésima iteração, possa-se retirar os valores para teste.

{\textcolor{CornflowerBlue}{void analyse\_river(int seed)}} é responsável por imprimir na tela os resultados do teste de acordo com o n-ésimo frame. Imprime características como fluxo do rio, altura da tela, largura da tela, zona de conforto do rio, entre outras características que são passados pelo usuário via parâmetro.

{\textcolor{CornflowerBlue}{static void analyse\_lines(TStrip strip)}} calcula e faz um relatório de determinada linha do rio, mostrando valores que não são passados pelo usuário, mas são gerados probabilisticamente, como a quantidade de água e terra, fluxo, e posição das margens em uma linha. 

\bigskip
\bigskip
\bigskip
\bigskip
\bigskip

{\Large \textcolor{NavyBlue}{ \textbf{Arquivos .c  }}}{\large \textcolor{CornflowerBlue}{  utils.c}}
\bigskip

Aloca memória com tamanho size\_t de forma que, caso haja um erro de alocação o programa é finalizado.


\end{document}
